\documentclass[12pt,a4paper]{article}
\usepackage[polish]{babel}
\usepackage[T1]{fontenc}
\usepackage[utf8x]{inputenc}
\usepackage{hyperref}
\usepackage{url}
\usepackage[]{algorithm2e}
\usepackage{listings}
\usepackage{graphicx}
\graphicspath{
    {media/}
}

\usepackage{color}
\usepackage{listings}

\lstloadlanguages{% Check Dokumentation for further languages ...
	C,
	C++,
	csh,
	Java
}

\definecolor{red}{rgb}{0.6,0,0} % for strings
\definecolor{blue}{rgb}{0,0,0.6}
\definecolor{green}{rgb}{0,0.8,0}
\definecolor{cyan}{rgb}{0.0,0.6,0.6}

\lstset{
	language=csh,
	basicstyle=\footnotesize\ttfamily,
	numbers=left,
	numberstyle=\tiny,
	numbersep=5pt,
	tabsize=2,
	extendedchars=true,
	breaklines=true,
	frame=b,
	stringstyle=\color{blue}\ttfamily,
	showspaces=false,
	showtabs=false,
	xleftmargin=17pt,
	framexleftmargin=17pt,
	framexrightmargin=5pt,
	framexbottommargin=4pt,
	commentstyle=\color{green},
	morecomment=[l]{//}, %use comment-line-style!
	morecomment=[s]{/*}{*/}, %for multiline comments
	showstringspaces=false,
	morekeywords={ abstract, event, new, struct,
		as, explicit, null, switch,
		base, extern, object, this,
		bool, false, operator, throw,
		break, finally, out, true,
		byte, fixed, override, try,
		case, float, params, typeof,
		catch, for, private, uint,
		char, foreach, protected, ulong,
		checked, goto, public, unchecked,
		class, if, readonly, unsafe,
		const, implicit, ref, ushort,
		continue, in, return, using,
		decimal, int, sbyte, virtual,
		default, interface, sealed, volatile,
		delegate, internal, short, void,
		do, is, sizeof, while,
		double, lock, stackalloc,
		else, long, static,
		enum, namespace, string},
	keywordstyle=\color{cyan},
	identifierstyle=\color{red},
}
\usepackage{caption}
\DeclareCaptionFont{white}{\color{white}}
\DeclareCaptionFormat{listing}{\colorbox{blue}{\parbox{\textwidth}{\hspace{15pt}#1#2#3}}}
\captionsetup[lstlisting]{format=listing,labelfont=white,textfont=white, singlelinecheck=false, margin=0pt, font={bf,footnotesize}}


\addtolength{\hoffset}{-1.5cm}
\addtolength{\marginparwidth}{-1.5cm}
\addtolength{\textwidth}{3cm}
\addtolength{\voffset}{-1cm}
\addtolength{\textheight}{2.5cm}
\setlength{\topmargin}{0cm}
\setlength{\headheight}{0cm}

\begin{document}
	
	\title{Modelowanie i analiza systemów informatycznych\\
		\bigskip
		\small{dokumentacja projektu \\
			System wspomagania decyzji}}
	\author{inż. Bartosz Ociepka\\
			inż. Beniamin Stecuła}
	\date{\today}
	
	\maketitle







	\newpage
\section*{Podział pracy}
	inż. Bartosz Ociepka - backend\\
	inż. Beniamin Stecuła - frontend
	
\section*{Udokumentowanie pracy}
	Dokumentowanie pracy odbyło się na kilka sposobów:
	\begin{itemize}
		\item utworzenie niniejszej dokumentacji,
		\item podział zadań na serwisie Trello:\\
				https://trello.com/b/se2oXQzD/polap-wspomaganie-decyzji
		\item przechowywanie kopii poprzednich wersji programu:\\
				https://github.com/BartoszOciepka/DiabetesNeuralNetwork
	\end{itemize}
	
\section*{Instrukcja obsługi}
	
\section*{Instrukcja wdrożenia}
	
\section*{Część testowa}
	
\section*{Baza danych}
\begin{figure}[h]
	\centering
	\centerline{
	\includegraphics[width=0.9\linewidth]{media/dbdiagram}
	}
	\caption{Diagram klas.}
	\label{fig:dbdiagram}
\end{figure}



\section*{Komponenty systemu}
	
\section*{Algorytm szyfrowania} 
	
\section*{Model systemu eksperckiego}

\section*{Przykładowy kod z aplikacji z testami}

\begin{lstlisting}
	Tutaj wklejamy pelen kod. 
\end{lstlisting}

\end{document}
